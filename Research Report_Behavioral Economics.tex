\documentclass[12pt, a4paper]{article}
\usepackage{blindtext}
\usepackage{geometry}
\geometry{a4paper,total={170mm,257mm},left=20mm,top=20mm}
\usepackage{natbib}
\usepackage{hyperref}
\hypersetup{colorlinks, citecolor=blue}
\usepackage{pifont}
\title {The Role of Nudge Theory in the Fight Against COVID-19 Pandemic}
\author{Vasavi}
\date{10\textsuperscript{th} May 2024}

\begin{document}
\maketitle

\vspace{75mm}
\textit{\textbf{Abstract}: Keeping the fruit at the eye level  counts as a nudge, banning junk food does not. In 2008, Richard Thaler and Cass Sustein published the book ‘Nudge: Improving decisions about health, wealth and happiness’ elaborating about how important it is to create a choice structure to nudge people to pick the right choice. The paper aims, through a review of the literature, to comprehend the significance of this nudge-theory in the battle against the COVID-19 pandemic, a public health issue of global concern.}

\break

\section{\centering Introduction}
The world battled an infectious disease epidemic affecting individuals in the most unprecedented ways and stressing out the governments to work out ways to prevent the disease from spreading and stopping further damage. The pandemic first started on 31st December 2019 when China alerted WHO of 27 cases of viral pneumonia in Wuhan. Subsequently, India Reported its first coronavirus case on 30th January 2020 in Thrissur district of Kerala. On the same day in a meeting, the Director-General of the World Health Organization (WHO) officially declared the novel coronavirus outbreak a Public Health Emergency of International Concern (PHEIC) as it had spread to 18 countries and was rapidly spreading all over the world (WHO Director-General's statement 2020).\\

On 24th March 2020, Prime Minister Shri. Narendra Modi, in his address to the nation declared nationwide lockdown and urged 1.3 billion people of the country to strictly follow the restrictions imposed. In his address to the nation he encouraged people to collectively fight against the pandemic and used the sentiment of ‘nationalism’ as a nudge to effectively measure and control the spread of the disease. This nudge along with many such nudges in the entire duration of the pandemic had positive public policy implications as the government managed to incentivise citizens to strictly abide by the guidelines and contain the spread of this highly contagious virus. 
Nudging is a design-based public policy approach which uses positive and negative reinforcements to modify the behavior of the people. Richard Thaler and Cass R Sustein, in their influential book, Nudge: Improving Decisions based on Health, Wealth and Happiness, defines nudge as any aspect of the choice architecture that alters people’s behavior in a predictable way without forbidding any options or significantly changing their economic incentives. To count as a mere nudge, the intervention must be easy and cheap to avoid. Putting the fruit at eye level counts as a nudge. Banning junk food does not.  This powerful tool of behavioral economics has been extensively used by the bureaucrats and governments to devise and implement public policies all over the world. Behavioral Economists engage in mapping the decision shortcuts that people use to effectively increase the decision making ability of the individuals. In the policy sphere, this has been efficiently used by policymakers by creating better ‘choice architecture’ while respecting the freedom of choice of the people. In other words, people can be incentivised in subtle ways towards choosing a better policy for themselves that provides cost-effective, positive and beneficial results while keeping the choices of the consumers the same.\\

This paper reviews the concept of Nudge Theory and analyzes the role of Nudge in the nation’s fight against the COVID-19 pandemic by reviewing various journal articles. Further, the paper draws its foundation from the study ‘India Nudges to contain COVID-19 pandemic: A reactive Public Policy analysis using machine learning based topic modeling’ by Ramit Debnath and Ronita Bardhan, where they investigated how government formed reactive policies to fight coronavirus across its policy sectors using Latent Dirichlet Allocation (LDA) algorithm, an unsupervised machine-based topic modeling. Apart from this, it also reviews the paper ‘Using Insights from Behavioral Economics to Mitigate the Spread of COVID‑19’ by Moslem Soofi, Farid Najafi and Behzad Karami Matin which introduces a number of insights from behavioral economics that help explain why people may behave irrationally during the COVID-19 pandemic. The paper further focuses on assessing the various public policies such as social distancing, hand wash techniques, online education and vaccination uptake etc used by the government during the pandemic through the lens of Nudge theory.\\

Moving forward, the paper builds on an experiment conducted using the control and treatment groups and substantiates the effectiveness of such policies, by replicating similar patterns of nudging under controlled environment, that helped the people of India to reduce the spread of the virus. \\


\section{\centering Literature Review}
As the country grappled with the highly infectious disease and the government announced complete lockdown from 25th March 2020, there were severe disruptions all over the world. All the businesses were shut down completely except extremely essential services like medical supplies, ration and PDS shops, grocery shops, fruits and vegetable markets and milk booths. Subsequently, the Covid-19 cases peaked in October 2020 and surpassed Brazil to record the second highest number of cases worldwide. In the address to the nation, the Prime Minister repeatedly emphasized the importance of social distancing and mask wearing, but the real issue that the country struggled with was the implementation of such measures at such a large scale, with huge disparities, with entire slums that had thousands of people using the same water source and bathrooms. Apart from this, one of the key issues was delivering the complicated medical jargon to the masses when many were illiterate.\\
Nudge provided the solutions to these problems. In order to implement social distancing measures across the country, the precautionary messages were spread through text messages and celebrity caller tunes. One such example is of using decorated and highly admired celebrity Amitabh Bachchan’s voice as caller tunes, spreading the precautionary message to the citizens across the country in their regional languages to nudge the citizens of the country to wear masks and maintain social distance. The Aarogya Setu app with its tracking feature became a contact tracing app that was used to assess one’s risk level and further encourage people to take better precautions. \citep{article1}


In West Bengal, India, a massive messaging initiative also investigated the significance of SMS. As reported in the paper "Messages on COVID-19 prevention in India Increased Symptoms Reporting and Adherence to Preventive Behaviors Among 25 Million Recipients with Similar Effects on Non-recipient Members of Their Communities," 25 million people received an SMS containing a 2.5-minute clip, which was delivered by 2019 Nobel laureate Abhijit Banerjee. \citep{article2}. One of the main ideas of nudge was emphasized in the study. Every message provided during this campaign had a motivational component that may affect oneself or everyone else nearby, which served as an encouraging feedback to nudge individuals to take action.\\
The paper also highlights that people in the communities who had not even received the message received it. This is significant since many people in India lack access to technology even in this day and age. Data being
distributed even in this way, would help propagate the word about adopting these life-saving practices due to herd behavior. Herd behavior is a concept that can be explained by the fact that people follow others' lead rather than gathering independent knowledge or coming to their own conclusions. \citep{article10}.\\
In order to encourage individuals to stay inside and limit the spread of the virus as much as possible, nudge theory was widely applied in this area also. Among the unique public policy nudges was the rerun of popular TV series from the 1980s and 1990s. It employed nostalgia as a nudge to get individuals to adopt the quarantine measures. As the paper by Debnath and Bardhan \citep{article10} mentions, these TV programs, which were aired on the national channel Doordarshan, varied from family entertainment to religious programs like the well-known Ramanand Sagar productions of the Ramayana and Mahabharata. Further, the paper also highlights the role of the government in using effective ways to nudge people to stay indoors by stopping fake news from spreading on social media. Fact-Checking Units (FCU) were established to encourage the public to double-check news sources. The Ministry of Human Resource Development exerted strong pressure on online learning (MHRD). The parents were then persistently nudged to support home schooling by promoting the usage of the GoI-funded National Digital Library of India. With a single-window search function, this digital resource offered access to a multilingual virtual archive of educational materials at various academic levels. The focus of policy was to use information and communication technologies to leverage online educational offerings. \citep{article10}.\\
Other two areas where nudge theory provided a very cost-effective and feasible public policy was hand washing and supply of essential items. The paper by Dhawan, Bhattacahrya and Mukherjee \citep{article1} highlights that several soap companies ran advertisements encouraging people to wash their hands. People were encouraged to follow this behavior by keeping sanitizers in public areas. According to a field experiment study conducted in India, peri-urban and rural households' hand washing improved after inexpensive soap dispensers were installed in their houses. \citep{article3}. In order to nudge citizens to stop the hoarding of essential supplies, extensive nudging was done to ensure that the government was actively involved in delivering essential items by engaging with the supply chain of Indian Railways. The transportation sector played a critical role in maintaining the supply chain of essential items. \citep{article1}. In this way, the government not only nudged the citizens to stop the hoarding of items but also made sure that the citizens strictly followed the social distancing measures.\\
The main argument of the Nudge is that governments can help people make better decisions while respecting their freedom of choice. This can be achieved by organizing the environment in which people make decisions—what Thaler and Sunstein \citep{article9} call choice architecture—for example, the cafeteria that promotes healthy eating by putting fruits first and desserts last.\\
From the standpoint of policy instruments, nudges are a less forceful form of government intervention than more conventional policy instruments like taxes and regulations, according to published research. Through the study by Debnath and Bardhan, the paper tries to understand the Government of India's use of nudges as a public policy tool in the fight against the coronavirus outbreak. In addition to this, the paper on ‘Applying Nudge to Public Health Policy: Practical Examples and Tips for Designing Nudge Interventions’ \citep{article4} mentions that the preventive healthcare, the provision of health and non-health services, long-term care/dementia prevention, community-based care systems, retirement planning, and technological innovation are just a few of the health policy domains and organizational activities where nudges can be implemented. Non-Communicable diseases (NCDs) are the main targets of Nudges. The World Health Organization (WHO) projected that by 2030, a seven-fold return might be obtained by investing in the most practical and cost-effective interventions to prevent and control non-communicable diseases (NCDs) in low- and middle-income countries. These "best-buy" therapies have the potential to lower alcohol and tobacco use, discourage bad eating habits, promote physical activity, control diabetes and cardiovascular disease, and manage cancer. Nudge interventions can have a significant role in the prevention and control of non-communicable diseases (NCDs) since most of these interventions depend on behavior modification. \citep{article4}. \\
Vaccine hesitancy and creating herd immunity was another sphere where nudge theory was strived to put to use. Citizens were nudged through SMS by the government of India and several government agencies. It has been demonstrated in two recent field tests that stressing "vaccine ownership" is beneficial for both the COVID-19 and flu vaccines. \citep{article5} discovered that the percentage of persons who received the flu shot rose when they saw the message "Full shot reserved for you." The statement "a COVID-19 vaccine has just been made available to you..." boosted the percentage of people who obtained the COVID-19 vaccine, according to research by \citep{article6}. The vaccines which were initially for elderly citizens and vulnerable groups were later rolled out for free for every citizen of the country. Apart from this, the vaccination centers were appropriately spread out across the city and country, to nudge people to take the vaccine. The paper on ‘Nudges for COVID-19 voluntary vaccination: How to explain peer information?’ \citep{article8} wonderfully aimed at discovering other-regarding information nudges that can reinforce people’s intention to receive the COVID-19 vaccine without impeding their autonomous decision-making. In case of vaccine hesitancy, the cognitive biases such as fear of removing organs on pretext covid vaccine which stem from misinformation and Social norms have a major impact on behavioral decisions that affect health-related behaviors like food habits, exercise habits, and immunization habits. For instance, neighbors' beliefs about vaccines can be a potent indicator of judgments made about acceptance or rejection of vaccines. \citep{article7}. To tackle these inefficiencies, several states have used nudge-based policies by incentivising people in very different ways. Numerous states have offered direct incentives via public and private organizations.\\
In Tamil Nadu's southern state, a Foundation with headquarters in Chennai offered lottery prizes, including gold coins, blenders, bikes, and washing machines, to individuals undertaking vaccinations. The Jan Swasthya Abhiyan of the state of Rajasthan proposes fully vaccinated families be given 150 days of MGNREGA work instead of the regular 100 days and 7 kg of grain per person per month instead of 5 kg. The Chandrapur Municipal Corporation in the state of Maharashtra has announced a vaccination bumper lucky draw, with lucrative prizes ranging from LED TVs, refrigerators and washing machines. \citep{article7}.\\ 
The paper on ‘Nudges for COVID-19 voluntary vaccination: How to explain peer information?’ \citep{article8} revisits the use of nudge theory in promoting vaccination across Japan for example before taking reservations for vaccinations, local governments in Japan mailed vouchers for free COVID-19 shots to citizens (albeit the flu shot costs money in Japan). People may feel that their vaccination was ensured as a result of this prior letter. \citep{article8}\\
In the context of this paper, several substantial bodies of research support the application of nudge theories to address such broad issues and encourage individuals by offering suitable options without impairing their autonomy. Historically, it has been assumed that nudges would only have the positive impact of assisting individuals in making decisions that align with their preferences, therefore improving their behavior for the benefit of both themselves and society. \citep{article8}.\\

\begin{center}
\section{Experiment Design}\end{center}
At such unprecedented times of global emergency as the governments were compelled to find extraordinary policies to safeguard its citizens, it utilized the concept of nudge theory to incentivise its citizens to contain the spread of the highly infectious coronavirus. The study aims to assess the efficacy of such nudge-theory based  interventions in incentivising the Indian population to adhere to the mitigation policies and actions undertaken.\\


\textbf{Research Design}: A randomized controlled trial (RCT) will be conducted to rigorously evaluate the effectiveness of Nudge Theory-based interventions in influencing people to follow COVID-19 guidelines and mitigation measures by dividing the respondents from diverse demographic backgrounds of the country into 2 groups based on their age. The treatment groups would be nudged through different ways based on their age groups and the control group will not be given any nudge. The data would be collected and analyzed using surveys and statistical tools.\\


\textbf{Informed Consent}: Before participating in the experiment, participants would be informed about the study’s goal, research goal, and potential risks or advantages and would then be asked to give their informed agreement to participate voluntarily.\\


\textbf{Participants}: Participants will be recruited from diverse demographic backgrounds across India. In addition to this a form would be rolled out across the country specifying the eligibility criteria for the experiment. Eligibility criteria include:\\
\begin{itemize}
\item Age: 18 years or older.
\item Ability to provide informed consent.
\item Access to a smartphone or computer with internet connectivity.
\end{itemize}
\textbf{Data Collection}: Data will be collected at both stages i.e. before and after both the groups receive nudges through an online survey that would include demographic questions as well as questions about behavioral change, attitudinal shift, how well the medical jargons were explained and the effectiveness of the nudges for post intervention.\\
For the pre-intervention, the survey would include asking them about their knowledge of covid 19(open ended) along with demographic questions and questions asking about whether they think their attitudes will change if they contract the virus and how effectively they would follow the covid guidelines.\\


\textbf{Treatment Groups}: the treatment group refers to the group that receives the experimental treatment, here we have segregated the participants into 2 groups based on their age as the separate age groups have different responses to different nudges.\\


\begin{enumerate}
\item \vspace{-6mm} \textbf{Young Adults(18-30 years old)}:\\
\begin{itemize}
\item[\ding{109}] \vspace{-6mm} Social Media Campaigns: Engaging influencers and celebrities to endorse preventive behaviors through visually appealing content on platforms like Instagram, TikTok, and Snapchat. The participants in this age group would be exposed to this content whereby influencers will be promoting covid prevention steps like mask wearing, hand-washing and social distancing.\\
\item[\ding{109}] \vspace{-5mm} Interactive Mobile Apps: Developing applications incentivizing adherence to preventive measures through virtual rewards and challenges like the Aarogya Setu app.\\
\item[\ding{109}] \vspace{-5mm} Motivation from institution: compulsory virtual guidance to be provided by the university or institution authorities to nudge students to follow the covid prevention measures.\\
\end{itemize}
\item \vspace{-5mm} \textbf{Middle Aged and Other Adults(31-60+ years)}:\\
\begin{itemize}
\item[\ding{109}] \vspace{-6mm} Television and Radio Advertisements: Broadcasting Public announcements featuring respected figures emphasizing mask-wearing, hand hygiene, and social distancing.\\
\item[\ding{109}] \vspace{-5mm} Printed Materials: Distributing informative pamphlets and posters in community centers and healthcare facilities, using clear graphics and concise messaging.
\item[\ding{109}] \vspace{0mm} Telephone Hotlines: Establishing toll-free helplines staffed by volunteers to provide personalized guidance on preventive measures and contact tracing the ones with symptoms.\\
\end{itemize}
\end{enumerate}
\vspace{-8mm} For vaccination uptake, both the age groups will be updated about the vaccination schedules available and the nearest vaccination center along with the messages busting myths around it\\

\textbf{Control Groups}: the control group refers to the group that does not receive the experimental treatment and hence will only be encouraged to follow standard Covid-19 guidelines.\\


\textbf{Data Analysis}: After receiving pre and post intervention surveys, the data will be analyzed using regression analysis, interactive graphs using data analysis tools like STATA, Power BI and Python to compare the results of the treatment and control groups to analyze pre and post intervention results for each group.\\



\textbf{Expected Outcome}: The expected outcome from the experiment would be:-\\
The treatment group, the group that received the nudge based intervention will demonstrate higher levels of adherence to Covid 19 restrictions. It will also be observed that the group of young adults will show greater responsiveness to social media based nudges and middle aged adults will show greater responsiveness to traditional media based nudges. The participants will have more knowledge about the Covid prevention strategies and will have a positive attitude towards the nudge based intervention. Adding to this, the constant updates regarding the vaccination status and busting vaccination myths via messages would enhance vaccination uptake of individuals immensely. As per the policy implication, if the results are found to be highly effective, the policy can be scaled up and implemented across the country.\\

\section{\centering Conclusion}
Overall, the nudge theory based interventions not only influenced individual behavior of the citizens by altering the choice structure but also encouraged collective action through herd behavior. In addition to this, the government nudged citizens to adopt behaviors that benefitted both themselves and the society at large. The application of nudge using diverse domains like education, transportation etc demonstrates the versatility of this behavioral science based concept and signifies its effectiveness in fighting such large scale issues.\\

The paper also highlights the importance of nudge theory to get rid of a major issue of vaccine hesitancy and educating the illiterate population. The findings from the RCT based experiment points out how nudges based on different age groups can be highly effective in encouraging citizens to adopt the covid appropriate behavior. \\

It is obvious that behavioral insights will continue to be crucial instruments for influencing public policy and advancing personal wellbeing as we sail through the ongoing difficulties brought on by the pandemic and beyond. Through the adoption of evidence-based strategies rooted in behavioral science, we can create communities that are healthier and more resilient going forward.\\

\break

\bibliographystyle{apalike}
\bibliography{sources}
\end{document}